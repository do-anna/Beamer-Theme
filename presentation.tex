\documentclass[handout]{beamer}
\usepackage[T1]{fontenc}
\usepackage[utf8]{inputenc}
\usepackage{tikz-cd,wrapfig}
\usepackage{tcolorbox}
\usepackage{listings}

%matematica
\usepackage{amsmath}
\usepackage{amssymb}
\usepackage{physics}
\usepackage{siunitx}




%For notes
%\usepackage{pgfpages}
%\setbeameroption{show notes on second screen}

%Typeset for sage
\lstdefinelanguage{Sage}[]{Python}
{morekeywords={False,sage,True},sensitive=true}
\lstset{
  frame=none,
  backgroundcolor=\color[RGB]{255,255,232},
  showtabs=False,
  showspaces=False,
  showstringspaces=False,
  commentstyle={\ttfamily\color{dgreencolor}},
  keywordstyle={\ttfamily\color{dbluecolor}\bfseries},
  stringstyle={\ttfamily\color{dgraycolor}\bfseries},
  language=Sage,
  basicstyle={\fontsize{10pt}{10pt}\ttfamily},
  aboveskip=0.3em,
  belowskip=0.1em,
  numbers=left,
  numberstyle=\footnotesize
}
\definecolor{dblackcolor}{rgb}{0.0,0.0,0.0}
\definecolor{dbluecolor}{rgb}{0.01,0.02,0.7}
\definecolor{dgreencolor}{rgb}{0.2,0.4,0.0}
\definecolor{dgraycolor}{rgb}{0.30,0.3,0.30}
\newcommand{\dblue}{\color{dbluecolor}\bf}
\newcommand{\dred}{\color{dredcolor}\bf}
\newcommand{\dblack}{\color{dblackcolor}\bf}

%%%%%

%Bibliography
\usepackage[style=numeric, maxnames=4,backend=bibtex]{biblatex}
% other styles: numeric authortitle
\addbibresource{biblio.bib}
\DeclareBibliographyCategory{fullcited} %for not citing in bibliography
\newcommand{\mybibexclude}[1]{\addtocategory{fullcited}{#1}} 
%%%%%

%Infos
\title{Measurement of the curvature of a mirror using a Michelson interferometer}
\author{\\\textbf{Candidate:} Anna Do'\\ \textbf{Supervisor:} Giacomo Lamporesi }
%\subtitle{}
\date{20 November 2024}
\institute{\\~\\Bachelor's Degree in Physics}
%\titlegraphic{...} 

%\usepackage{giacomo}
%Theme settings
%\usetheme{metropolis}
\usetheme[physics]{beunitn}
\usecolortheme{rose}
\setbeamercovered{dynamic}



% Theorem definitons 
\theoremstyle{plain}
\newtheorem{teo}{Teorema}[section]
\newtheorem{lem}[teo]{Lemma}
\newtheorem{prop}[teo]{Proposizione}
\newtheorem{cor}[teo]{Corollario}
\newtheorem*{form}{Formula}

\theoremstyle{remark}
\newtheorem{rem}{Osservazione}
\newtheorem{rems}[rem]{Remarks}

\theoremstyle{definition}
\newtheorem{deff}[teo]{Definizione}
\newtheorem{idea}{Idea}
\newtheorem*{nota}{Notazione}

% Commands 
\newcommand{\sage}{\href{https://www.sagemath.org}{\includegraphics[height=\fontcharht\font`\B]{../images/sage.png} }}
\newcommand{\jupyter}{\href{https://www.sagemath.org}{\includegraphics[height=\fontcharht\font`\B]{../images/jupyter.png} }}
\newcommand{\noqed}{\let\qed\relax}
\newcommand{\PS}{\mathcal{P}_S}
\newcommand{\Z}{\mathbb{Z}}
\newcommand{\F}{\mathbb{F}}
\newcommand{\K}{\mathbb{K}}
\newcommand{\ZZ}[1]{\mathbb{Z}_{#1}}
\newcommand{\Q}{\mathbb{Q}}
\newcommand{\C}{\mathbb{C}}
\newcommand{\R}{\mathbb{R}}

\DeclareMathOperator*{\eqb }{=}
\DeclareMathOperator{\Gal}{Gal}
\DeclareMathOperator{\ord}{ord}
\definecolor{lgreencolor}{rgb}{0.43,0.612,0.435} % Verde chiaro


\newcommand{\mybox}[1]{\fcolorbox{lgreencolor}{lgreencolor}{#1}}


\begin{document}

	\begin{frame}[plain]
	    \maketitle
	\end{frame}



%%%%%%%%%%%%%%%%%%%%%%%%%%%%%%%%%%%%%%%%%%%%%%%%%%%%%%%%%%%%%%%
\begin{frame}{Introduction}
\begin{columns}[T]
\begin{column}{0.5\textwidth}
  
  %This thesis is about an experiment whose aim is to analyse a method for correcting distortions in the spatial mode that may be present in the beam of 
  %the laser of Virgo GW interferomete


  The topics covered in this presentation are:
  \begin{itemize}
    \item Michelson interferometer
    \item Two theoretical models: geometric optics and interference of Gaussian beam
    \item Data analysis
    \item Phase shift method
  \end{itemize}
\end{column}
\begin{column}{0.5\textwidth}
\begin{figure}[h!]
  \centering
  \includegraphics[width = 1\linewidth]{../latex/Immagini/foto set up/foto1.jpg}
\end{figure}
\end{column}
\end{columns}

\end{frame}
%%%%%%%%%%%%%%%%%%%%%%%%%%%%%%%%%%%%%%%%%%%%%%%%%%%%%%%%%%%%%%%
  \begin{frame}{Michelson interferometer/interference of light}

    \begin{columns}
      \begin{column}{0.5\textwidth}
        The electric field can be expressed in the complex notation as:
        \begin{equation*}
          \vec{E}=\vec{E_0}e^{i\Phi}
        \end{equation*}
      If $\vec{E_1}$ and $\vec{E_2}$ are the electric in the two arms, the total 
        filed is $\vec{E_{tot}}=\vec{E_1}+\vec{E_2}$, then the intensity on the screen is proportional to:
        \begin{equation*}
          I\propto<|\vec{E_{tot}}|^2>=2 I_0^2 [ 1+ \cos (\Phi_1-\Phi_2) ]=2 I_0^2 [ 1+ \cos(k\Delta z)]
        \end{equation*}

 
      \end{column}

      \begin{column}{0.5\textwidth}
        \begin{figure}[h!]
          \centering
          \includegraphics[width = 1\linewidth]{../latex/Immagini/3Dmickelson.jpg}
        \end{figure}
      \end{column}
    \end{columns}
   
  \end{frame}
  %%%%%%%%%%%%%%%%%%%%%%%%%%%%%%%%%%%%%%%%%%%%%%%%%%%%%%%%%%%%
  \begin{frame}{Experimental set-up}
  \vspace{-10pt}
    \begin{figure}[h!]
      \centering
          \centering
          \includegraphics[width = 0.9\linewidth]{../latex/Immagini/setup.png}
  \end{figure}
  \end{frame}
  %%%%%%%%%%%%%%%%%%%%%%%%%%%%%%%%%%%%%%%%%%%%%%%%%%%%%%%%%%%%%%%%	
    \begin{frame}{Experimental set-up}
      \begin{figure}[h!]
        \centering
        \includegraphics[width = 0.8\linewidth]{../latex/Immagini/foto set up/foto2.jpg}
      \end{figure}
    \end{frame}

  %%%%%%%%%%%%%%%%%%%%%%%%%%%%%%%%%%%%%%%%%%%%%%%%%%%%%%%%%%%%%%%%	
  \begin{frame}{Geometric optics}
    \begin{figure}[h!]
      \centering
      \includegraphics[width = 0.7\linewidth]{../latex/Immagini/geometrical_optics.jpg}
    \end{figure}
  The optical path difference in the two arms of the interferometer is:
    {\small
    \begin{align*}
      \Delta z&=2d -(d+\overline{NM}+\overline{ML})=\frac{2(1-\cos\theta)[d+d\cos\theta+R\cos^2\theta]}{2cos^2\theta-1}\simeq\\
      ~&\\
        &\simeq\frac{r^2}{R^2}(2d+R)\simeq\frac{r^2}{R} \quad \rightarrow \quad \mybox{$I\propto\left[ 1+ \cos\left( \frac{k r^2}{R}\right)\right]$}
      \end{align*}
      
    }

  \end{frame}
  %%%%%%%%%%%%%%%%%%%%%%%%%%%%%%%%%%%%%%%%%%%%%%%%%%%
  \begin{frame}{Gaussian beams}
    The equation of a Gaussian beam is:
    \begin{equation*}
      E= E_{0} \frac{w_{0}}{w(z)}\exp\left(\frac{-r^2}{w^2(z)}\right)\exp\left[ik\left(z+\frac{r^2}{2R(z)}\right)-i\tan^{-1}\left(\frac{z}{z_{0}}\right)\right]
    \end{equation*}
    where:
    \begin{equation*}
      z_0=\frac{\pi w_0^2}{\lambda}\qquad w(z)=w_0\sqrt{1+\left(\frac{z}{z_0}\right)^2}\qquad R(z)=z\left[1+\left(\frac{z_0}{z}\right)^2\right]
    \end{equation*}

    Let us call  $\text{Beam}_1$ the beam reflected by the spherical mirror and $\text{Beam}_2$ the beam coming from the planar mirror:
    \begin{tabular}{c|c|c|c|c}
      ~&$w_0$ & $z_0$& $w(z)$ & $R(z)$\\
      \hline
      $\text{Beam}_1$&$\simeq 0.025~\unit{\milli\meter}$& $\simeq 3.8 ~\unit{\milli\meter}$& $w_{01}z_1/z_{01}$& $z_1$\\
      $\text{Beam}_2$&$\simeq 5~\unit{\milli\meter}$& $\simeq 150 ~\unit{\meter}$& $w_{02}$& $z^2_{02}/z_2$\\
  \end{tabular}

  \end{frame}
  %%%%%%%%%%%%%%%%%%%%%%%%%%%%%%%%%%%%
  \begin{frame}{Gaussian beams}

    The intensity of the electric magnetic field on the screen will be: 
   {\footnotesize
    \begin{align*}
      &I\propto|E_{tot}|^2=\\
      &~\\
      &=E_0^2 \left\{ \exp \left( -\frac{2r^2}{w_{02}^2} \right) + \left( \frac{z_{01}}{z_1} \right)^2 \exp \left( -\frac{2r^2 z_{01}^2}{w_{01}^2 z_1^2} \right)+\right. \\
      &\left. + \left( \frac{z_{01}}{z_1} \right) \exp \left[ -r^2 \left( \frac{1}{w_{02}^2} + \left(\frac{z_{01}}{w_{01} z_1}\right)^2 \right) \right]
      2 \cos \left[ k(z_2 - z_1) + \frac{k r^2}{2} \left( \frac{z_2}{z_{02}^2} + \frac{1}{z_1} \right) \right] \right\}\simeq\\
      &~\\
      &\simeq  E_0^2 \left[ \exp \left( -\frac{2r^2}{w_{02}^2} \right) + \left( \frac{z_{01}}{z_1} \right)^2 + 
      2 \left( \frac{z_{01}}{z_1} \right)^2 \exp \left( -\frac{r^2}{w_{02}^2} \right) \mybox{$\cos \left( \frac{k r^2}{R} \right)$} ~\right]
    \end{align*}
   }

    
  \end{frame}

  %%%%%%%%%%%%%%%%%%%%%%%%%%%%%%%%%%%%%%%%%%%%%%
  \begin{frame}{Data analysis}
    The fit function is:
    \begin{equation*}
    I=A\left[ B^2 + \exp(-2ar^2)+2 B \exp(-ar^2)\cos(br^2+c)\right]
    \end{equation*}

    \begin{columns}
      \begin{column}{0.5\textwidth}
      where from the previous equation we have that: 
      {\footnotesize
      \begin{align*}
       a&=\frac{1}{w_{02}^2} \qquad b=\frac{2\pi}{\lambda (R+2d)} \\
       ~\\
       B&=\frac{z_{01}}{z_1} \\
      \end{align*}
      }
 
      \end{column}

      \begin{column}{0.5\textwidth}
        \begin{figure}[h!]
          \centering
          \begin{minipage}{1\textwidth}
            \centering
            \includegraphics[width = 1\linewidth, trim = 340pt 0pt 180pt 0pt , clip]{../latex/Immagini/data/cerchi.png}
          \end{minipage}
        \end{figure}
      \end{column}
    \end{columns}
    
  \end{frame}

  %%%%%%%%%%%%%%%%%%%%%%%%%%%%%%%%%%%%%%%
  \begin{frame}{Data analysis}
     
    \begin{figure}[h!]
      \centering
      \begin{minipage}{0.8\textwidth}
        \centering
        \includegraphics[width = \linewidth]{../latex/Immagini/data/fit.png}
    \end{minipage}
   
    \end{figure}


    \begin{equation*}
      R=\frac{2\pi}{b\lambda}=(2.57\pm0.01)~\unit{\meter}
    \end{equation*}

  \end{frame}


  %%%%%%%%%%%%%%%%%%%%%%%%%%%%%%%%%%%%%
  \begin{frame}{Phase shift interferometry}

    Let us consider the intensity on the camera at different position of the planar mirror ($d+\delta \Phi_z$):
    \begin{equation*}
      I=A(x,y)\Big[1+B(x,y)\cos \Big(\Phi(x,y)+\delta \Phi_z \Big)\Big]+off(x,y)
    \end{equation*}

  The surface of the mirror $h(x,y)$ is related to the phase $\Phi(x,y)$:
  \begin{equation*}
    h(x,y)=\frac{\lambda}{4\pi} \Phi(x,y)
  \end{equation*}


  For fixed value of $r$, we observe that the intensity is a sinusoidal function of $\delta \Phi_z $:
  \begin{equation*}
    I(\delta \Phi_z )=A \sin(\delta \Phi_z +\Phi)+off=A \sin(n v \Delta t +\Phi)+off
  \end{equation*}

  \end{frame}

  %%%%%%%%%%%%%%%%%%%%%%%%%%
  \begin{frame}{Phase shift interferometry}
      \begin{figure}[h!]
        \centering
        \begin{minipage}{0.49\textwidth}
            \centering
            \includegraphics[width = \linewidth]{../latex/Immagini/phase shift/x0_y0.png}
        \end{minipage}
        \begin{minipage}{0.49\textwidth}
            \centering
            \includegraphics[width = \linewidth]{../latex/Immagini/phase shift/x20_y50.png}
        \end{minipage}
        \caption{The first image is for the pixel with coordinates (0,0) and the second for the pixel with coordinates (20,50).}
      \end{figure}
            

  \end{frame}

%%%%%%%%%%%%%%%%%%%%%%%%%%%%%%%%%%%%%%%%%%%%%%%%%%%%%%%%%%%%%%%%%%%%%%%%%%%%%%%%%%%%%%
  \begin{frame}{Phase shift interferometry}
    Analysis of the data performed by the code:

    \begin{itemize}
      \item[$\bullet$] get $\Phi$ from the data, there are two option:
      \begin{itemize}
        \item[$\circ$] sinusoidal fit
        \item[$\circ$] fft : fast Fourier transform
      \end{itemize}
      
      \item[$\bullet$] unwrap the phase $\Phi$

      \item[$\bullet$] fit to get the parameter that describe the surface, there are two option:
      \begin{itemize}
        \item[$\circ$] {\tiny $\left[A \Big((y-y_c) \cos(\alpha) +(x-x_c) \sin(\alpha)\Big)^2 + B\Big((x-x_c)  \cos(\alpha) - (y-y_c)\sin(\alpha)\Big)^2\right] + \text{{o}} $ }
        \item[$\circ$] Zernike polynomials
      \end{itemize}
    \end{itemize}
    
  \end{frame}

\end{document}