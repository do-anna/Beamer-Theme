\documentclass{beamer}

\usepackage{lipsum}
\usepackage{todonotes}

%Macro for solving path problem in theme calling
\makeatletter
  \def\beamer@calltheme#1#2#3{%
    \def\beamer@themelist{#2}
    \@for\beamer@themename:=\beamer@themelist\do
    {\usepackage[{#1}]{\beamer@themelocation/#3\beamer@themename}}}

  \def\usefolder#1{
    \def\beamer@themelocation{#1}
  }
  \def\beamer@themelocation{}
  
\graphicspath{{../../}}

%todo: ovviamente, questo andrà cambiato
%\usepackage{../../beamerthemebeunitn}

\usefolder{../../}
\usetheme{beunitn}

\usepackage[italian]{babel}
\usepackage[utf8]{inputenc}

%Bibliography
\usepackage[style=numeric, maxnames=4,backend=bibtex]{biblatex}
% other styles: numeric authortitle
\addbibresource{biblio.bib}

\title{Un breve test}
\subtitle{esempio di applicazione del template, adesso allungo il titolo per vedere come viene}
\author{autori vari}
\institute{Università di Trento}
\date{30 febbraio 2042}

\begin{document}
% c'è un overfull hbox, ma tanto è solo un dummy test

\begin{frame}[fragile]
  \titlepage
\end{frame}

\begin{frame}{Contenuti}
  \tableofcontents
\end{frame}

\section{Titolo sezione numero 1}

\begin{frame}{Questo è un frame}
    \begin{block}{Un blocco di testo importante}
        \lipsum[66]
    \end{block}
    \pause
    \begin{itemize}
      \item \textbf{un punto}: posso scrivere cose
      \pause
      \item \textbf{altro punto}: posso scriverle due volte
    \end{itemize}
\end{frame}

\section{Altro titolo sezione}
\subsection{Sottosezione}
\begin{frame}{Questo è un frame dal titolo decisamente lungo}
    \textbf{Dal vangelo secondo Lipsum:} \\
    \pause
    \textit{\lipsum[75]} \\ 
    (Lipsum, 75)
\end{frame}

\subsection{Seconda sottosezione}

\begin{frame}{Formule}
    \[
    \mathcal L_{\mathcal T}(\vec{\lambda})
    = \sum_{(\mathbf{x},\mathbf{s})\in \mathcal T}
       \log P(\mathbf{s}\mid\mathbf{x}) - \sum_{i=1}^m
       \frac{\lambda_i^2}{2\sigma^2}
    \]
\end{frame}

\begin{frame}{Formule 2}
    \begin{align*}
        S(\omega) 
        &= \frac{\alpha g^2}{\omega^5} e^{[ -0.74\bigl\{\frac{\omega U_\omega         19.5}{g}\bigr\}^{\!-4}\,]} \\
        &= \frac{\alpha g^2}{\omega^5} \exp\Bigl[ -0.74\Bigl\{\frac{\omega       U_\omega 19.5}{g}\Bigr\}^{\!-4}\,\Bigr] 
    \end{align*}
\end{frame}


\begin{frame}{Il problema delle immagini}
    \missingfigure[figwidth=6cm]{immagine a caso}
\end{frame}

\begin{frame}{Conclusione}
    Iniziamo con un alertblock per attirare l'attenzione \dots
    \begin{alertblock}{Una frase profonda e filosofica}
        \lipsum[66]
    \end{alertblock}
    \beamerskipbutton{Bottone finto}
\end{frame}
\begin{frame}{}
    \dots
    ed ora un exampleblock per finire in bellezza:
    \begin{exampleblock}{Problemi di sonno?}
        "Chi ben mangia, dorme la sera." 
        \flushright
        (Orazio, forse)
    \end{exampleblock}
\end{frame}

\begin{frame}
	\nocite{CF} Come ha detto il caro \cite{RAM} \newline
	\pause
	\fullcite{SIN}
\end{frame}

\section{Sezione finale}

\begin{frame}{Bibliografia}
	\printbibliography[heading=none]
\end{frame}

\begin{frame}{Table of contents}
	\tableofcontents
\end{frame}

\end{document}


