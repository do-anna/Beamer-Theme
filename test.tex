\documentclass{beamer}

\usepackage{lipsum}
\usepackage{todonotes}

%todo: ovviamente, questo andrà cambiato.
\mode<presentation>
{
  \usetheme{Luebeck}      
  \usecolortheme{orchid} 
  \usefonttheme{structuresmallcapsserif}  
  \setbeamertemplate{navigation symbols}{}
  \setbeamertemplate{caption}[numbered]
} 

\usepackage[italian]{babel}
\usepackage[utf8]{inputenc}

\AtBeginSection[]{
  \begin{frame}
  \vfill
  \centering
  \begin{beamercolorbox}[sep=8pt,center,shadow=true,rounded=true]{title}
    \usebeamerfont{title}\insertsectionhead\par%
  \end{beamercolorbox}
  \vfill
  \end{frame}
}

\title{Un breve test}
\subtitle{esempio di applicazione del template}
\author{autori vari}
\institute{Università di Trento}
\date{30 febbraio 2042}

\begin{document}

\begin{frame}
  \titlepage
\end{frame}

\begin{frame}{Contenuti}
  \tableofcontents
\end{frame}

\section{Titolo sezione}

\begin{frame}{Questo è un frame}
    \begin{block}{Un blocco di testo importante}
        \lipsum[66]
    \end{block}
    \pause
    \begin{itemize}
      \item \textbf{un punto}: posso scrivere cose
      \pause
      \item \textbf{altro punto}: posso scriverle due volte
    \end{itemize}
\end{frame}

\section{Altro titolo sezione}
\subsection{Sottosezione}

\begin{frame}{}
    \textbf{Dal vangelo secondo Lipsum:} \\
    \pause
    \textit{\lipsum[75]} \\ 
    (Lipsum, 75)
\end{frame}

\subsection{Seconda sottosezione}

\begin{frame}{Formule}
    \[
    \mathcal L_{\mathcal T}(\vec{\lambda})
    = \sum_{(\mathbf{x},\mathbf{s})\in \mathcal T}
       \log P(\mathbf{s}\mid\mathbf{x}) - \sum_{i=1}^m
       \frac{\lambda_i^2}{2\sigma^2}
    \]
\end{frame}

\begin{frame}{Formule 2}
    \begin{align*}
        S(\omega) 
        &= \frac{\alpha g^2}{\omega^5} e^{[ -0.74\bigl\{\frac{\omega U_\omega         19.5}{g}\bigr\}^{\!-4}\,]} \\
        &= \frac{\alpha g^2}{\omega^5} \exp\Bigl[ -0.74\Bigl\{\frac{\omega       U_\omega 19.5}{g}\Bigr\}^{\!-4}\,\Bigr] 
    \end{align*}
\end{frame}


\begin{frame}{Il problema delle immagini}
    \missingfigure[figwidth=6cm]{immagine a caso}
\end{frame}

\begin{frame}{Conclusione}
    \begin{alertblock}{Una frase profonda e filosofica}
        \lipsum[66]
    \end{alertblock}
\end{frame}

\end{document}


